
\documentclass[a4paper,12pt,single,pdftex]{scrartcl}
%\usepackage[ngerman]{babel}
\usepackage{color}
\usepackage{amsmath}
\usepackage{times}
\usepackage{graphicx}
\usepackage{fancyheadings}
\usepackage{hyperref}
\setlength{\parindent}{0.6pt}
\setlength{\parskip}{0.6pt}
\title{}
 

\begin{document} 
\maketitle
\newpage
%\tableofcontents
\newpage

\section{Datasets}
\subsection{M\_D09Q1}

    
      {\bf Surface reflectance of R and NIR at 250m each 8-days}.
    \\

    
      From this, we can calculate NDVI.
    \\

  \subsection{M\_D11A2}

    
      {\bf LST day and night at 1000m every 8-days.}
    \\

    
      This is an input for VPD following the simple method of Hashimoto 2008
    \\

  \subsection{MCD15A2H}

    
      {\bf LAI / FPAR at 500m every 8-day. }
    \\

    
      R. Myneni, Y. K., T.Park. (2015). MCD15A3H MODIS/Terra+Aqua Leaf Area Index/FPAR 4-day L4 Global 500m SIN Grid V006. NASA EOSDIS Land Processes DAAC. https://doi.org/10.5067/MODIS/MCD15A3H.006
    \\

  \subsection{M\_17AH2}

    
      {\bf GPP \& Net Photosynthesis at 500m every 8-days.}
    \\

    
      S. Running, Q. M., M.Zhao. (2015). MOD17A2H MODIS/Terra Gross Primary Productivity 8-Day L4 Global 500m SIN Grid V006. NASA EOSDIS Land Processes DAAC. https://doi.org/10.5067/MODIS/MOD17A2H.006
    \\

  \subsection{CHIRPS-V2}

    
      {\bf Precipitation, prec. anomaly and prec. z-score at 5000m daily.}
    \\

    
      Funk, Chris, Pete Peterson, Martin Landsfeld, Diego Pedreros, James Verdin, Shraddhanand Shukla, Gregory Husak, James Rowland, Laura Harrison, Andrew Hoell \& Joel Michaelsen. "The climate hazards infrared precipitation with stations—a new environmental record for monitoring extremes". Scientific Data 2, 150066. doi:10.1038/sdata.2015.66 2015.
    \\

  \subsection{ESI}

    
      
        
          {\bf Evaporative Stress Index (ESI) at 4000m weekly}
        \\

        
          Anderson, M.C., C. Hain, B. Wardlow, A. Pimstein, J.R. Mecikalski and W.P. Kustas, 2011:Evaluation of drought indices based on thermal remote sensing of evapotranspiration over the continental United States. Journal of Climate, 24(8):2025–2044.
        \\

      
    
  \section{Possible datasets}


JPL VPD\section{Questions}

    
      {\bf 1. Is it possible to detect drought early? Or, how early can we detect drought?}
    \\

    \begin{itemize}
  \item 
        some variables hopefully show sensitivity earlier, so we can detect the drought onset.
      
    \end{itemize}
  
  
    
      {\bf 2. Model(s) derived from the 1st objective can correlate well to the yield? How was the panorama of drought during the studied period?}
    \\

    \begin{itemize}
  \item 
        A comprehensive assessment of the drought occurrences throughout the years.
      \item 
        A long and thoughtful backward variables selection will tell me what matters what don't in a yield prediction model.
      
    \end{itemize}
  
  
    
      {\bf 3. Is LUE affected by drought?}
    \\

    \begin{itemize}
  \item 
        
          My hyphothesis for that is: if GPP = PAR x fPAR x LUE or GPP = APAR x LUE, is it an immediate relation? Is LUE constant regardless the stress? I need better insights here.
        \\

      \item 
        
          How is VPD related to drought?
        \\

      
    \end{itemize}
  
  \section{Study area and period}
\subsection{Brazil}
\subsubsection{Soybean in Parana and Mato Grosso.}


Yield data for modeling is an issue.
We might have to figure out how to break down the data by rotation.
It is time to work on contacts and communication with EMBRAPAs, IAPAR, Fundacao Santo Andre, IMEA and so on.\subsubsection{Maize as second crop?}


For some regions, if the yearly mapping is accurate, it will be easy to break down yield data to maize and soy.\subsection{USA}


Maize in Nebraska, Illinois, Iowa.

Soybean somewhere\section{Outputs}
\subsection{NDVI time-series}


from M\_D09Q1 >Vegetation indices may be submitted to a filtering procedure\subsection{GPP time-series}


from M\_17AH2 > Filtered or not?\subsection{LAI / FPAR}


from MCD15AH2 > Just as input for LUE calculation? FPAR = APAR/PARinc\subsection{LUE}

    
      {\bf from LUE = GPP/(FPAR x PARinc); however, there is no PAR! }
    \\

    
      {\bf an alternative for LUE is PRI (Photochemical Reflectance Index)}.
    \\

    
      Drolet, G. G., Middleton, E. M., Huemmrich, K. F., Hall, F. G., Amiro, B. D., Barr, A. G., … Margolis, H. A. (2008). Regional mapping of gross light-use efficiency using MODIS spectral indices. Remote Sensing of Environment, 112(6), 3064–3078. https://doi.org/10.1016/j.rse.2008.03.002
    \\

  \subsection{LST time-series}


as input for VPD following Hashimoto 2008.\subsection{VPD}


I'm not sure wether 8-day time-step will tell me anything about how VPD is driven by drought and, hence, vegetation underdevelopment.\subsection{SPI3}


This is calculated in R using SPEI package, by applying the function over the data-frames.\subsection{ESI}


From ESI > weekly evapotranspiration anomalies.
My preliminary studies for QuickDRI have showed that ESI has earlier response to drought up to two weeks.\section{People}

    
      {\bf Fabio Ricardo Marin (ESALQ)}
    \\

    
      fabio.marin@usp.br
    \\

    
      p: +55 19 3447 8507
    \\

  \subsection{
    
      {\bf Daniel de Castro Victoria EMBRAPA-InA}
    \\

    
      daniel.victoria@embrapa.br
    \\

  }


Mato Grosso good data
    
      {\bf Jeslyn Christopher Brown (KU)}
    \\

    
      jcbrown2@ku.edu
    \\

    
      p: 785 864 5543
    \\

    
      Brian's friend!
    \\

  
    
      {\bf Martha Anderson (USDA-ARS)}
    \\

    
      martha.anderson@ars.usda.gov
    \\

    
      p: 301 504 6616
    \\

  

\end{document}
